\documentclass{article}
\usepackage{physics}
\usepackage{amssymb}
\usepackage{amsmath}
\usepackage{mathtools}
\usepackage{hyperref}
\usepackage{xepersian}
\settextfont{Yas}

\author{مرتضی شکرانه \\ \\ راهنما: دکتر اجتهادی}
\title{گزارش تسک اول پروژه}
\date{\today}
\begin{document}
\maketitle	
\section{مقدمه} 
ساده ترین مدلی که برای حرکت یک کرم می توان در نظر داشت، مدلِ مهره-فنر می باشد. طوری که بدنه ی یک کرم، مار و امثالهم را می توان به نقطه های بسیار زیادی تقسیم کرد که به یک دیگر متصل شده اند. نتیجه ی دینامیک این اجزا، در مقیاس بزرگتر، یک حرکت کرم وار و یا دست کم، یک دینامیک جایگزیده ی تقربیا پیوسته است. 
در ادامه برای تشریحِ جنبه های مختلف این ایده و با نگاه به هدفِ پروژه که شبیه سازی پلیمر ها می باشد تلاش هایی را انجام دادم.     
\section{دینامیک براونی و مدل Rouse} 
یک دینامیکِ مناسب برای حرکت اجزای کرم یا به زبان بهتر، مونومرها در یک پلیمر خطی ساده، دینامیک براونی می باشد. که حرکت ذرات به وسیله ی یک نیروی رندم(با توزیع یکنواخت) تعیین می شود. برای جزئیات این دینامیک، در notebook jupyter برنامه ای نوشتم که می توانید آن را مشاهده کنید.
\footnote{برای دسترسی راحت تر به گزارش ها و همینطور کدها، یک ریپازیتوری گیت ایجاد کردم که از \href{https://github.com/morteza-shokraneh/My-MD}{ایـــــــــنجا} می توانید ببینید.}
\\برای شبیه سازی یک پلیمر خطی ساده و بررسی دینامیک و تغییر شکل آن به علت حرکت براونی، طبق مدل Rouse ، از ماژول HOOMD استفاده کردم که یک ابزار بسیار جذاب برای کارهای شبیه سازی است و متود های مفیدی دارد. در نهایت با ست کردن پارامترها و یک پیکربندی مناسب برای سیستم و استفاده از متود براونی برای تحول زمانی، کار شبیه سازی انجام شد. همچنین کمیت های مورد ملاحظه در فایل دت محاسبه شده اند. چندین گیف هم که به کمک برنامه ovito از فایل .gsd تولید شده توسط ماژول هوم دی متصور کرده ام نیز به عنوان نتایج تیپیکالِ تحول زمانیِ پلیمر در ریپازیتوری موجود هستند.   
\section{مقالات و کتاب ها} 
کتاب ها و مقالات زیادی در زمینه ی شبیه سازی پلیمرها و دینامیک مولکولی وجود دارد.
هنوز چارچوب مشخصی برای من در این زمینه شکل نگرفته است. فضایی را در گوگل درایو ایجاد خواهم کرد که به صورت مستمر و هفتگی مقالاتی را که می خوانم در آن جا قرار دهم که اگر در آینده کار به مرحله ی چاپ کردنِ نتایج و مقاله رسید، دایرکتوری منسجمی برای ارجاعات وجود داشته باشد.  
\section{تسک های آینده و چشم انداز} 
\section{تلاش های به اتمام نرسیده!}
• برنامه ی شی گرا برای تولید فایلِ داده لمپس \\
• یک کار فان برای شبیه سازی حرکت کرم با استفاده از analysis gait
\\

\end{document}